\chapter{Feasibility}
\section{Introduction}

\par
PlanADay Application: Suggest One Day Trip is a mobile application that helps people who
want to go on a one-day trip without wasting time to create the plan that has many steps to do
such as searching for places, available hours on each place, finding the route path and so on. The
application will help reduce the time spent on organizing a one-day trip by suggesting the place
due to the user’s preferences and providing the place detail
\section{Problem Statement}
\par
Planning a one-day trip can be challenging and time-consuming. It involves multiple steps,
including searching for places of interest, checking operating hours, determining suitable routes,
and aligning schedules. Many people lack the time or abilities to plan an efficient trip, which
often results in frustration, inefficiency, or missed opportunities to explore exciting places.
Current solutions may offer individual recommendations or maps, but they do not take a
comprehensive approach to creating a personalized, structured plan. This gap highlights the need
for an application like PlanADay, which streamlines the process by providing personalized
recommendations to help users maximize their travel experience with minimal effort.
\section{Related research and projects}
\noindent
\textbf{Location based services: ongoing evolution and research agenda} \\
\noindent
Explores the evolution and research trends of Location-Based Services (LBS) in the context of
the mobile information era, highlighting their significance in delivering location-dependent
information to users. It identifies key research challenges, including advancements in
positioning, modeling, communication, and the evaluation of LBS-generated data. Additionally,
the article addresses social, ethical, and behavioral considerations associated with LBS
integration into daily life, offering a comprehensive research agenda to shape the future of LBS
in a positive and impactful way.
\newpage
\noindent
\textbf{Personalized Day-Trip Planning: A TSP-TW-Based Multimodal Multicriteria Optimisation
Approach} \\
Innovative approach to creating personalized one-day travel itineraries by enhancing the
Traveling Salesman Problem with Time Windows (TSP-TW). It incorporates multi-criteria
optimization, flexible activities, park-and-ride options, and multiple transport modes to address
diverse mobility preferences. Using choice experiments, the authors derive utility functions to
optimize itineraries, achieving a 16.19% improvement in travel utility compared to plans focused
solely on travel time. The method was validated through simulations in a medium-sized German
city, showcasing its effectiveness in real-world scenarios.
\section{Requirement specifications for the new system}
\subsection{Function requirements}
\begin{enumerate}
	\item Preference setting: Users should be able to select their own preference of
	categories for the application to provide the suggestion for the user.
	\item User Input: The application should allow users to input trip details with plan
	name, categories of interested place, location area, start time, start date, and the
	number of places to visit.
	\item Location suggestion: The application should suggest places based on user
	preferences and user input to find the appropriate place.
	\item Itinerary Generation: The system must provide a sequential trip plan with details
	like order of visit, travel time between locations, and routing path to each place.
	\item Plan Customization: Users should be able to customize the plan including edit
	plan name. Delete or add more places, reorder the place, and regenerate the plan.
	\item Attraction Details: The application should display detailed information for each
	attraction, including photos, descriptions, rating, and opening hours.
\end{enumerate}

\subsection{Data requirements}
\begin{enumerate}
	\item Location data from Google Map API
\end{enumerate}
\subsection{System requirements}
\begin{enumerate}
	\item Smartphone with Android OS or iOS
	\item Has geo-positioning sensor (GPS)
\end{enumerate}
\newpage
\section{Implementation techniques}
\subsection{Frontend}
\begin{itemize}
	\item Language
	\begin{itemize}
		\item Dart
	\end{itemize}
\end{itemize}
\begin{itemize}
	\item Framework
	\begin{itemize}
		\item Flutter
	\end{itemize}
\end{itemize}
\subsection{Backend}
\begin{itemize}
	\item Language
	\begin{itemize}
		\item Typescript
	\end{itemize}
\end{itemize}
\begin{itemize}
	\item Framework
	\begin{itemize}
		\item Bun
		\item Elysia
	\end{itemize}
	\item External Service
	\begin{itemize}
		\item GoogleMap API
	\end{itemize}
	\item Testing
	\begin{itemize}
		\item Jest
		\item Postman
	\end{itemize}
\end{itemize}
\subsection{Infrastructure}
\begin{itemize}
	\item Operating System
	\begin{itemize}
		\item Debian Linux
	\end{itemize}
	\item Hosting
	\begin{itemize}
		\item SSD Node (Virtual Private Server)
	\end{itemize}
	\item Container Management
	\begin{itemize}
		\item Docker
	\end{itemize}
	\item Image Storage
	\begin{itemize}
		\item Docker Registry
	\end{itemize}
	\item External Service
	\begin{itemize}
		\item Google Cloud Platform
	\end{itemize}
	\item Web-server
	\begin{itemize}
		\item Nginx
	\end{itemize}
	\item Database
	\begin{itemize}
		\item mongoDB
		\item Redis
		\item PostgreSQL
	\end{itemize}
\end{itemize}
\subsection{Other}
\begin{itemize}
	\item Editor Development: Visual Studio Code
	\item DevOps: Github Action
	\item Project Management: Trello
	\item Online meetings: Microsoft Teams and Discord
	\item Design: Figma and Canva
	\item Clipart and Icon: Freepik, Scale by Flexiple, Flaticon and Feather
	\item Diagram Design: Lucid Chart
\end{itemize}
\newpage
\section{Implementation plan}