\chapter{Summary and suggestions}
\section{Introduction}
This chapter provides a summary of the project, which comprises three parts: project summary, problems encountered and solutions, and suggestions for further development. The first section summarizes and provides the overall results of the project. The second section outlines the problems encountered and proposes solutions to the limitations of the project. Finally, suggestions are provided for the future improvement of the project.

\section{Project summary}
TravelKit is a mobile application that assists individuals in navigating through public transportation options like buses, sky trains, subways, boats, and mini trucks to reach their desired destinations. Users can access information such as pricing, travel duration, and the number of transfers involved in their chosen route. Additionally, the application includes a feature enabling users to share their routes with one another. From the project’s objective and scope, we can create a software that can calculate travel routes, distances, and estimate travel costs with optimizing the suggested routes based on user personal choices.

\section{Problems encountered and solutions}
There are two main problems that we encountered during the project. First problem that we find difficult to cope with is the process of developing mobile application that uses Dart with Flutter as a programming language and framework that we are not quite familiar with. So, we need to learn some new programming styles for the development. The second challenge involves incorporating Neo4j as a graph database, a tool with which we lack prior implementation experience. Consequently, we must familiarize ourselves with the database's query language in order to effectively build our system.

\newpage
\section{Suggestions for further development}
\subsection{Integration of AI in our system}
Currently, our application has a feature that suggest routes based on the users preferences but we can use artificial intelligence to enhance this feature by collecting data from user. We can use that data to predict what sorting preferences user is likely to choose. Furthermore, we can suggest more accurate recommended places to user if we have a proper ai model.
\subsection{Responsive design and implementation}
Currently, our application is working correctly with only certain devices. To support wide variety of mobile and desktop screens, it need revisions in design and implementation to support  the responsive design.
\subsection{High coupling of the data}
Currently, our service has accurate data that we can return route data correctly, but to modify route data, we need to update both in the MongoDB and Neo4j separately which is consequence of manually inserting data to all databases and all relation in Neo4j needs to changed for the updated route.
