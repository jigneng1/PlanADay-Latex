\chapter{Summary and suggestions}
\section{Project Sumary}
PlanADay Application: Personalized One-Day Trip Planning System is a mobile application that
assists users in creating custom itineraries through three key types of suggestions: based on user
interests, you might be interested, and location-based recommendations. The definitions of each
suggestion are explained below.\vspace{1cm} \\
\textbf{Plan Customization:} \\
The system recommends locations that match the user’s stated preferences, such as cafes, parks,
restaurants, museums, gyms, shopping, etc. Using the user's input, including starting location,
preferred starting date and time, and desired number of stops. PlanADay generates a complete
itinerary. This feature ensures that the trip plan aligns closely with the user’s interests while
exploring nearby attractions that they have not visited before. \vspace{1cm} \\
\textbf{Completed Plans Shared by Others:} \\
This suggestion feature analyzes the user’s interests and bookmarked plans from other users to
recommend similar places they might enjoy. For instance, if a user sets their interests as cafes
and shopping, the system suggests completed plans shared by other users in the application that
align with these interests. This helps users discover new places and plans while ensuring the
recommendations remain relevant to their preferences.\par
With these features, PlanADay enables users to explore new destinations, enjoy tailored
itineraries, and make the most out of their one-day trips. It provides a seamless blend of
personalization, discovery, and flexibility.
\newpage

\section{Problems encountered and solutions}
We encountered some problems during the mobile application development phase, and some
issues took time and effort to solve. The problems encountered and solutions will discuss as
follow:
\subsection{Location fetching issue}
In our system design, we rely on the Google Maps API to fetch location data, as it provides a
comprehensive and reliable source of geolocation information worldwide. However, during
implementation, we encountered a significant challenge: the cost associated with using the
Google Cloud Platform (GCP). GCP requires users to subscribe and potentially incur charges to
access their resources, including the Google Maps API. Since cost-efficiency is a priority for us,
we explored ways to minimize expenses while maintaining functionality. After thorough
research, we discovered that Google Cloud Platform offers a \$300 free credit to new users. This
credit can be utilized without being charged unless the account is upgraded to a fully paid
subscription. By strategically registering and using this free credit, we temporarily mitigated the
cost barrier. This approach allows us to proceed with development and testing without incurring
immediate expenses. However, this solution is only temporary, and we recognize the need to plan
for long-term sustainability by either optimizing API usage or exploring alternative solutions.
Furthermore, to minimize API call costs, we are considering strategies caching frequently
accessed location data and reducing redundant API calls. These measures aim to reduce
dependency on external APIs and limit the associated expenses.

\subsection{Data storage}
The main challenge with storing location data is managing the large volume of information
fetched during the planning process. Continuously saving every fetched or edited location to the
database would consume excessive storage and increase costs unnecessarily. To address this, we
use a temporary cache to store fetched location data. Users can edit, add, or delete locations in
the cache during the planning phase. Only the finalized plan, once submitted by the user, is
stored in the database. This approach minimizes storage usage, reduces database overhead, and
ensures only relevant data is saved. The cache is also periodically cleared to free up memory.

\newpage
\begin{table}[]
    \centering
    \renewcommand{\arraystretch}{1.2} % Adjust row height for better readability
    \begin{tabular}{|p{2.2in}|p{4in}|}
    \hline
    \rowcolor[HTML]{C0C0C0} 
    \multicolumn{1}{|c|}{\cellcolor[HTML]{C0C0C0}\textbf{Suggestion}}                                                                                       & \multicolumn{1}{c|}{\cellcolor[HTML]{C0C0C0}\textbf{Details}}   \\ \hline
    Authentication & Enhance the sign-in and sign-up processes by integrating Gmail validation, moving beyond the basic username and password approach. This feature ensures a more secure and reliable authentication process for users. Additionally, incorporate third-party authentication options, such as Google, Facebook, and other social media platforms, to simplify the login process and improve user convenience and make account management more accessible and efficient. \\ \hline
    Place Pool Expansion & Efficient Filtering Algorithms: Use advanced filtering techniques to optimize the matching process and make the most of the existing pool before resorting to real-time fetching. Along with dynamic data fetching, implement a fallback mechanism to fetch additional data in real-time when the pool does not meet user requirements. This ensures that users always receive relevant suggestions. \\ \hline
    Details and more options & Enhance the detailed view of each place by providing cost estimates and more specific filters. For example, food-related locations can include details such as whether they are halal, vegan, or vegetarian-friendly. These additions offer users more clarity and enable them to customize their preferences more effectively. \\ \hline
    Recommendation Plan from others & Introduce an interactive interface for users to engage with trip plans recommended by others. Before publishing their plans, users can add comments, share experiences, and provide feedback on the plans they followed, fostering collaboration and building a more engaging community within the application. \\ \hline
    Frontend \& Backend Code quality & Apply other higher-efficiency techniques to build the large-scale recommendation system both frontend and backend. \\ \hline
    \end{tabular}
    \caption{List of Suggestions for further development}
    \label{tab:suggestion-table}
\end{table}


